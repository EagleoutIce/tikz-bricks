\documentclass{article}
\usepackage{tikz-bricks}

\begin{document}
Hier ein Beispiel:\bigskip

\begin{tikzpicture}[line join=round]
    \BrickSetup
    % optional argument marks fields to be put in foreground
    % you are able to use buffers
    \BrickSixXTwo[5/1,6/1]{0,0,0}{cyan}
    \BrickSixXTwo{0,0,2}{orange}
    \BrickSixXTwo[3/1,4/1]{0,0,4}{cyan}

    \BrickOneXSix[1/5]{6,0,4}{red!65!black}
    \BrickFourXOne{3,\h,-1}{yellow!35!brown}

    \BrickTwoXFour{2,\h,3}{green!85!teal}
\end{tikzpicture}

Eine weitere Demo:

\resizebox{\linewidth}{!}{\begin{tikzpicture}[line join=round]
    \BrickSetup
\BrickOneXFour{-1,0,0}{teal}
\BrickOneXFour{-1,0,4}{teal}
\BrickOneXFour{-1,0,8}{teal}
\BrickOneXFour{-1,0,12}{teal}

\BrickOneXFour{0,0,0}{red}

\BrickOneXFour{3,0,0}{green}
\BrickOneXFour{0,1.2025,0}{green}
\BrickOneXFour{1,1.2025,0}{red}
\BrickOneXFour{2,1.2025,0}{green}
\BrickOneXFour{3,1.2025,0}{red}
\BrickOneXFour{1,0,4}{green}
\BrickOneXFour{2,0,4}{red}
\BrickFourXOne{0,0,9}{orange}
\BrickFourXOne{0,0,10}{orange}
\BrickFourXOne{0,0,11}{orange}
\BrickFourXOne{0,0,12}{orange}
\BrickFourXOne{0,2*1.2025,0}{orange}

\foreach \xyi in {0,4,8,12} {
    \BrickOneXFour{4,0,\xyi}{teal}
}
\BrickOneXFour{0,1.2025,12}{green}

\BrickTwoXFour{11,1.2025,12}{green}
\BrickOneXFour{11,2*1.2025,12}{red}
\BrickOneXFour{12,2*1.2025,12}{teal}

\BrickTwoXTwo{12,2*1.2025,5}{blue!50!white}
\BrickFourXOne{-5,2*1.2025,12}{red}
\end{tikzpicture}}
\end{document}